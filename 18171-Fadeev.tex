\documentclass[a4 paper, 12 pt]{extarticle}
\usepackage{titlesec}
\titleformat{\section}{\normalfont\normalsize\bfseries}{\thesection}{1em}{}
\usepackage[utf8]{inputenc}
\usepackage[T2A]{fontenc}
\usepackage[russian]{babel}
\usepackage{amsfonts}
\usepackage{mathtext}
\usepackage{amsthm}
\usepackage{amsmath}
\usepackage{amssymb}
\usepackage{ragged2e}
\usepackage[unicode, pdftex]{hyperref}
\usepackage{upgreek}
\usepackage[left=3cm, right=1cm, top=1.5cm, bottom=2cm, bindingoffset=0cm]{geometry}
\usepackage{indentfirst}
\usepackage[pdftex]{graphicx}
\usepackage{enumerate}
\usepackage{tocloft}
\usepackage{textcomp}
\usepackage{setspace}

%\renewcommand{\thesection}{\arabic{section}.}

\renewcommand{\P}[1]{\mathbb{P} \left( #1 \right)}
\newcommand{\E}[1]{\mathbb{E} \left[ #1 \right]}
\newcommand{\R}{\mathbb{R}}
\newcommand{\tail}[1]{\overline{#1}}
\newcommand{\cip}{\overset{p}{\to}}
\newcommand{\Zi}{Z_{\infty}}
\setcounter{page}{1}

\newtheorem{lemma}{Лемма}[section]
\newtheorem{theorem}{Теорема}[section]
\newtheorem{remark}{Замечание}
\newtheorem{corollary}{Следствие}
\newtheorem{proposition}{Утверждение}[section]
\newtheorem{example}{Пример}

\doublespacing

\title{Задача рассеяния для уравнения Шрёдингера на метрических графах}
\author{С. А. Фадеев}
\date{\today}

\begin{document}
	\begin{singlespacing} % На титульнике одинарный интервал
		\thispagestyle{empty} % Не нумеруем титульный лист
		\begin{center}
			\normalsize{МИНИСТЕРСТВО НАУКИ И ВЫСШЕГО ОБРАЗОВАНИЯ РОССИЙСКОЙ ФЕДЕРАЦИИ}\\
			\hfill \break
			\normalsize{ФЕДЕРАЛЬНОЕ ГОСУДАРСТВЕННОЕ АВТОНОМНОЕ ОБРАЗОВАТЕЛЬНОЕ УЧРЕЖДЕНИЕ ВЫСШЕГО ОБРАЗОВАНИЯ}\\
			\hfill \break
			\normalsize{<<НОВОСИБИРСКИЙ НАЦИОНАЛЬНЫЙ ИССЛЕДОВАТЕЛЬСКИЙ ГОСУДАРСТВЕННЫЙ УНИВЕРСИТЕТ>>}
			\normalsize{(Новосибирский государственный университет, НГУ)}\\ 
			\hfill \break
			\large{\textbf{Механико-математический факультет}}\\
			\hfill\break
			\large{Кафедра теории функций}\\
			\hfill \break
			\large{Направление подготовки <<Математика>>, магистратура}\\
			\hfill\break
			\normalsize{\textbf{ВЫПУСКНАЯ КВАЛИФИКАЦИОННАЯ РАБОТА МАГИСТРА}}\\
			\hfill \break
			\large{Фадеева Степана Александровича}\\
			\hfill \break
			\large{\textbf{Задача рассеяния для уравнения Шрёдингера на метрических графах}}\\
			\hfill \break
		\end{center}
	\end{singlespacing}
	
	\noindent
	\begin{minipage}[t]{80mm}\parindent=0em
		\large{\textbf{<<К защите допущен>>}\par
			Заведующий кафедрой,\par
			д.ф.-м.н.,проф.\par
			\makebox[3cm]{\large{Медных А.Д.}}/\makebox[3cm]{\hrulefill}\par}
		\small{~(Фамилия, И.О.)~/~ ~(подпись, МП)}\par
		\large{«\makebox[1cm]{\hrulefill}»\makebox[4cm]{\hrulefill}2020 г.}
	\end{minipage}
	\hfill
	\begin{minipage}[t]{80mm}\parindent=1.4em
		\large{\textbf{<<Научный руководитель>>}\par
			к.ф.-м.н.\par
			с.н.с. ИМ им. Соболева\par
			\makebox[3cm]{\large{Дедок В.А.}}/\makebox[3cm]{\hrulefill}\par}
		\small{~(Фамилия, И.О.)~/~ ~(подпись, МП)}\par
		\large{«\makebox[1cm]{\hrulefill}»\makebox[4cm]{\hrulefill}2020 г.}
	\end{minipage}
	
	\hfill \break
	\begin{flushright}
		\large{Дата защиты:\large{<<\makebox[1cm]{\hrulefill}>>\makebox[4cm]{\hrulefill}2020 г.}}
	\end{flushright}
	
	\hfill \break
	\begin{singlespacing}
		\begin{center}
			\large{\textbf{\\Новосибирск,~2020}}
		\end{center} 
	\end{singlespacing}

	\section*{РЕФЕРАТ}
	{\it Название работы:} Задача рассеяния для уравнения Шрёдингера на метрических графах.
	
	{\it Количество страниц:} ??.
	
	{\it Количество используемых источников:} ??.
	
	{\it Ключевые слова}: задача рассеяния, уравнение Шрёдингера, метрические графы, компактные графы, законы Кирхгофа, квантовые графы, спектральная хирургия.
	
	Объект исследования --- рассеяние на метрических графах, прямые и обратные задачи. Цели дипломной работы ~--- исследование изменений данных рассеяния при топологических преобразованиях квантовых графов, а также изучение зависимости данных рассеяния от рассеивающих потенциалов. 	Исследование основано на технике спектральной хирургии квантовых графов, последовательном приближении рассеивающего потенциала ступенчатыми функциями, отслеживании изменения данных рассеяния при подобных приближениях. Область применения и рекомендации ~--- математическое моделирование мезоскопических систем, наноэлектроника, теория квантового хаоса, квантовые вычисления. В результате исследования были обобщены теоремы о спектральной хирургии квантовых графов, а также изучена сходимость данных рассеяния при константных возмущениях потенциала.
	
	\newpage
	\tableofcontents
	\newpage
	\section*{ВВЕДЕНИЕ}
	\addcontentsline{toc}{section}{ВВЕДЕНИЕ}
	Задача распространения волн в ветвящихся структурах и сетях играет важную роль во многих областях современной физики \cite{TransparentQuantumGraphs}. Для моделирования подобных явлений часто используются квантовые графы: компактные графы с добавленными бесконечными дугами. На каждой дуге определено одномерное стационарное уравнение Шрёдингера, в качестве граничных условий в узлах используются законы Кирхгофа для квантовых графов \cite{KirchhoffRule}. Важной особенностью рассматриваемых конструкций является распространение волн вдоль рёбер, сопровождающееся рассеянием (отражением)  волны в узлах графа. Бесконечно накапливаясь, отражённые волны могут оказать существенное влияние на решение задачи рассеяния.
	
	Впервые квантовые графы одномерных нитевидных структур, соединённых в вершинах, появились в работе H.~Kuhn \cite{Kuhn} и использовались для моделирования свободных электронов органических молекул. В данной работе органическая молекула была представлена набором атомов, расположенных в определённых координатах, соединённых связями, вдоль которых электроны подчиняются одномерному стационарному уравнению Шрёдингера с подходящим потенциалом.
	
	Позднее аналогичный подход был применён в задачах моделирования сверхпроводников  и искусственных материалов \cite{Superconductivity}, теории мезоскопических систем \cite{Kowal}, теории квантового хаоса \cite{QuantumChaos}.
	
	Наряду с задачами рассеяния на некомпактных графах, многие исследователи изучают спектр энергий системы уравнений Шрёдингера на компактных графах (Хотелось бы сослаться на статьи Курасова и компании за 2019 год, но не совсем понятно, что же там происходит. С одной стороны, рассматривают квантовые графы и delta-couplings, с другой стороны, оценивают спектр энергий. Но ведь он определён только для компактного графа!)
	
	Нас интересуют некомпактные графы, содержащие полубесконечные дуги. Такая гибкая конструкция позволяет воспроизводить черты многомерных объектов, будучи одномерной \cite{GerasimenkoPavlov}. (? что значит "будучи одномерной"? У нас ведь система уравнений!)
	
	В силу топологической сложности графов, в общем случае поиск решения задачи рассеяния является достаточно сложной проблемой. Методы решения, применяемые к одномерным и многомерным задачам рассеяния, не могут быть применимы к данной задаче. Поэтому для решения задачи рассеяния на топологически сложном графе предлагается найти решение на топологически примитивных графах, а затем склеить друг с другом примитивы согласно некоторым правилам преобразования данных рассеяния. Этот метод называется техникой спектральной хирургии квантовых графов и описан в \cite{SpectralSurgery}.
	
	Данная работа состоит из двух частей. 
	
	Первая часть посвящена обобщению теорем спектральной хирургии. Получен результат о склейке двух графов, содержащих неизвестную компактную часть, известные данные рассеяния. В дальнейшем планируется обобщить полученные результаты для получения теоремы о склейке по произвольному количеству дуг полубесконечной длины.
	
	Вторая часть посвящена исследованию зависимости данных рассеяния на некомпактном графе от рассеивающих потенциалов. В силу сложности задачи рассеяния, были рассмотрены некоторые частные случаи графов и потенциалов на внутренних дугах. Для данных частных случаев графов (петля и кольцо с полубесконечными дугами) были решены прямые задачи по определению данных рассеяния для различных потенциалов. Были исследованы свойства данных рассеяния на таких графах со ступенчатыми потенциалами на внутренних ребрах, численно изучены эффекты возникновения новых волн при трансформации потенциала. В дальнейшем планируется применить полученные результаты для построения вычислительно эффективного метода решения обратной задачи рассеяния.
	
	
	\newpage
	\section{ОДНОМЕРНАЯ ЗАДАЧА РАССЕЯНИЯ}
\newpage
\addcontentsline{toc}{section}{ЛИТЕРАТУРА}
\renewcommand{\refname}{ЛИТЕРАТУРА}
\begin{thebibliography}{1} % Ссылки на литературу приводятся только на английском языке (даже если у источника нет англоязычной версии).
	
	\bibitem{TransparentQuantumGraphs} J.~R.~Yusupov, K.~K.~Sabirov, M.~Ehrhardt, D.~U.~Matrasulov, {\it Transparent Quantum Graphs}, Phys. Lett. A 383, 2382 (2019).
	
	\bibitem{KirchhoffRule} V.~Kostrykin, R.~Schrader, {\it Kirchhoff's Rule for Quantum Wires} J. Phys. A: Math. Gen. Vol.32, (1999), 595--630.
	
	\bibitem{Kuhn} H.~Kuhn, {\it A Quantum-Mechanical Theory of Light Absorption of Organic Dyes and Similar Compounds}, J. Chem. Phys. \textbf{17} (1949), https://doi.org/10.1063/1.1747143.
	
	\bibitem{Superconductivity} S.~Alexander, {\it Superconductivity of networks. A percolation approach to the effects of disorder}, Phys. Rev. B \textbf{27}, 1541 (1983).
	
	\bibitem{Kowal} D.~Kowal, U.~Sivan, O.~Entin-Wohlman, Y.~Imry, {\it Transmission through multiply-connected wire systems}, Phys. Rev. B \textbf{42}, 9009 (1990).
	
	\bibitem{QuantumChaos} T.~Kottos, U.~Smilansky, {\it Quantum chaos on graphs}, Phys. Rev. Lett. \textbf{79}, 4794--7 (1997).
	
	\bibitem{GerasimenkoPavlov} Н.~И.~Герасименко, Б.~С.~Павлов, {\it Задача рассеяния на некомпактных графах}, ТМФ, \textbf{74}:3 (1988), 345 -- 359.
	
	\bibitem{SpectralSurgery} А.~Н.~Бондаренко, В.~А.~Дедок, {\it Спектральная хирургия квантовых графов}, Сиб. журн. индустр. матем., \textbf{7}:4 (2004), 16--28.

	
\end{thebibliography}
\end{document}
