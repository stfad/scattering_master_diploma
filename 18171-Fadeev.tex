\documentclass[a4 paper, 12 pt]{extarticle}
\usepackage{titlesec}
\titleformat{\section}{\normalfont\normalsize\bfseries}{\thesection}{1em}{}
\usepackage[utf8]{inputenc}
\usepackage[T2A]{fontenc}
\usepackage[russian]{babel}
\usepackage{amsfonts}
\usepackage{mathtext}
\usepackage{amsthm}
\usepackage{amsmath}
\usepackage{amssymb}
\usepackage{ragged2e}
\usepackage[unicode, pdftex]{hyperref}
\usepackage{upgreek}
\usepackage[left=3cm, right=1cm, top=1.5cm, bottom=2cm, bindingoffset=0cm]{geometry}
\usepackage{indentfirst}
\usepackage[pdftex]{graphicx}
\usepackage{enumerate}
\usepackage{tocloft}
\usepackage{textcomp}
\usepackage{setspace}

%\renewcommand{\thesection}{\arabic{section}.}

\renewcommand{\P}[1]{\mathbb{P} \left( #1 \right)}
\newcommand{\E}[1]{\mathbb{E} \left[ #1 \right]}
\newcommand{\R}{\mathbb{R}}
\newcommand{\tail}[1]{\overline{#1}}
\newcommand{\cip}{\overset{p}{\to}}
\newcommand{\Zi}{Z_{\infty}}
\setcounter{page}{1}

\newtheorem{lemma}{Лемма}[section]
\newtheorem{theorem}{Теорема}[section]
\newtheorem{remark}{Замечание}
\newtheorem{corollary}{Следствие}
\newtheorem{proposition}{Утверждение}[section]
\newtheorem{example}{Пример}
\newtheorem{Def}{Определение}[section]

\newcommand{\RomanNumeralCaps}[1]
{\MakeUppercase{\romannumeral #1}}

\doublespacing

\title{Задача рассеяния для уравнения Шрёдингера на метрических графах}
\author{С. А. Фадеев}
\date{\today}

\begin{document}
	\begin{singlespacing} % На титульнике одинарный интервал
		\thispagestyle{empty} % Не нумеруем титульный лист
		\begin{center}
			\normalsize{МИНИСТЕРСТВО НАУКИ И ВЫСШЕГО ОБРАЗОВАНИЯ РОССИЙСКОЙ ФЕДЕРАЦИИ}\\
			\hfill \break
			\normalsize{ФЕДЕРАЛЬНОЕ ГОСУДАРСТВЕННОЕ АВТОНОМНОЕ ОБРАЗОВАТЕЛЬНОЕ УЧРЕЖДЕНИЕ ВЫСШЕГО ОБРАЗОВАНИЯ}\\
			\hfill \break
			\normalsize{<<НОВОСИБИРСКИЙ НАЦИОНАЛЬНЫЙ ИССЛЕДОВАТЕЛЬСКИЙ ГОСУДАРСТВЕННЫЙ УНИВЕРСИТЕТ>>}
			\normalsize{(Новосибирский государственный университет, НГУ)}\\ 
			\hfill \break
			\large{\textbf{Механико-математический факультет}}\\
			\hfill\break
			\large{Кафедра теории функций}\\
			\hfill \break
			\large{Направление подготовки <<Математика>>, магистратура}\\
			\hfill\break
			\normalsize{\textbf{ВЫПУСКНАЯ КВАЛИФИКАЦИОННАЯ РАБОТА МАГИСТРА}}\\
			\hfill \break
			\large{Фадеева Степана Александровича}\\
			\hfill \break
			\large{\textbf{Задача рассеяния для уравнения Шрёдингера на метрических графах}}\\
			\hfill \break
		\end{center}
	\end{singlespacing}
	
	\noindent
	\begin{minipage}[t]{80mm}\parindent=0em
		\large{\textbf{<<К защите допущен>>}\par
			Заведующий кафедрой,\par
			д.ф.-м.н.,проф.\par
			\makebox[3cm]{\large{Медных А.Д.}}/\makebox[3cm]{\hrulefill}\par}
		\small{~(Фамилия, И.О.)~/~ ~(подпись, МП)}\par
		\large{«\makebox[1cm]{\hrulefill}»\makebox[4cm]{\hrulefill}2020 г.}
	\end{minipage}
	\hfill
	\begin{minipage}[t]{80mm}\parindent=1.4em
		\large{\textbf{<<Научный руководитель>>}\par
			к.ф.-м.н.\par
			с.н.с. ИМ им. Соболева\par
			\makebox[3cm]{\large{Дедок В.А.}}/\makebox[3cm]{\hrulefill}\par}
		\small{~(Фамилия, И.О.)~/~ ~(подпись, МП)}\par
		\large{«\makebox[1cm]{\hrulefill}»\makebox[4cm]{\hrulefill}2020 г.}
	\end{minipage}
	
	\hfill \break
	\begin{flushright}
		\large{Дата защиты:\large{<<\makebox[1cm]{\hrulefill}>>\makebox[4cm]{\hrulefill}2020 г.}}
	\end{flushright}
	
	\hfill \break
	\begin{singlespacing}
		\begin{center}
			\large{\textbf{\\Новосибирск,~2020}}
		\end{center} 
	\end{singlespacing}

	\section*{РЕФЕРАТ}
	{\it Название работы:} Задача рассеяния для уравнения Шрёдингера на метрических графах.
	
	{\it Количество страниц:} ??.
	
	{\it Количество используемых источников:} ??.
	
	{\it Ключевые слова}: задача рассеяния, уравнение Шрёдингера, метрические графы, компактные графы, законы Кирхгофа, квантовые графы, спектральная хирургия.
	
	Объект исследования --- рассеяние на метрических графах, прямые и обратные задачи. Цели дипломной работы ~--- исследование изменений данных рассеяния при топологических преобразованиях квантовых графов, а также изучение зависимости данных рассеяния от рассеивающих потенциалов. 	Исследование основано на технике спектральной хирургии квантовых графов, последовательном приближении рассеивающего потенциала ступенчатыми функциями, отслеживании изменения данных рассеяния при подобных приближениях. Область применения и рекомендации ~--- математическое моделирование мезоскопических систем, наноэлектроника, теория квантового хаоса, квантовые вычисления. В результате исследования были обобщены теоремы о спектральной хирургии квантовых графов, а также изучена сходимость данных рассеяния при константных возмущениях потенциала.
	
	\newpage
	\tableofcontents
	\newpage
	\section*{ВВЕДЕНИЕ}
	\addcontentsline{toc}{section}{ВВЕДЕНИЕ}
	Задача распространения волн в ветвящихся структурах и сетях играет важную роль во многих областях современной физики \cite{TransparentQuantumGraphs}. Для моделирования подобных явлений часто используются квантовые графы: компактные графы с добавленными бесконечными дугами. На каждой дуге определено одномерное стационарное уравнение Шрёдингера, в качестве граничных условий в узлах используются законы Кирхгофа для квантовых графов \cite{KirchhoffRule}. Важной особенностью рассматриваемых конструкций является распространение волн вдоль рёбер, сопровождающееся рассеянием (отражением)  волны в узлах графа. Бесконечно накапливаясь, отражённые волны могут оказать существенное влияние на решение задачи рассеяния.
	
	Впервые квантовые графы одномерных нитевидных структур, соединённых в вершинах, появились в работе H.~Kuhn \cite{Kuhn} и использовались для моделирования свободных электронов органических молекул. В данной работе органическая молекула была представлена набором атомов, расположенных в определённых координатах, соединённых связями, вдоль которых электроны подчиняются одномерному стационарному уравнению Шрёдингера с подходящим потенциалом.
	
	Позднее аналогичный подход был применён в задачах моделирования сверхпроводников  и искусственных материалов \cite{Superconductivity}, теории мезоскопических систем \cite{Kowal}, теории квантового хаоса \cite{QuantumChaos}.
	
	В настоящее время ведутся исследования в области спектральных оценок квантовых графов \cite{SpectralEstimates}, в том числе при использовании обобщённых граничных условий (delta couplings) \cite{deltaCouplings}. Кроме того, рассматриваются времязависимые модели в электромагнитных полях \cite{Popov1, Popov2}.
	
	Нас интересуют некомпактные графы, содержащие полубесконечные дуги. Такая гибкая конструкция позволяет воспроизводить черты многомерных объектов, будучи одномерной \cite{GerasimenkoPavlov}.
	
	В силу топологической сложности графов, в общем случае поиск решения задачи рассеяния является достаточно сложной проблемой. Методы решения, применяемые к одномерным и многомерным задачам рассеяния, не могут быть применимы к данной задаче. Поэтому для решения задачи рассеяния на топологически сложном графе предлагается найти решение на топологически примитивных графах, а затем склеить друг с другом примитивы согласно некоторым правилам преобразования данных рассеяния. Этот метод называется техникой спектральной хирургии квантовых графов и описан в \cite{SpectralSurgery}.
	
	Данная работа состоит из двух частей. 
	
	Первая часть посвящена обобщению теорем спектральной хирургии. Получен результат о склейке двух графов, содержащих неизвестную компактную часть, известные данные рассеяния. В дальнейшем планируется обобщить полученные результаты для получения теоремы о склейке по произвольному количеству дуг полубесконечной длины.
	
	Вторая часть посвящена исследованию зависимости данных рассеяния на некомпактном графе от рассеивающих потенциалов. В силу сложности задачи рассеяния, были рассмотрены некоторые частные случаи графов и потенциалов на внутренних дугах. Для данных частных случаев графов (петля и кольцо с полубесконечными дугами) были решены прямые задачи по определению данных рассеяния для различных потенциалов. Были исследованы свойства данных рассеяния на таких графах со ступенчатыми потенциалами на внутренних ребрах, численно изучены эффекты возникновения новых волн при трансформации потенциала. В дальнейшем планируется применить полученные результаты для построения вычислительно эффективного метода решения обратной задачи рассеяния.
	
	
	\newpage
	\section{ОДНОМЕРНАЯ ЗАДАЧА РАССЕЯНИЯ}
	\subsection{ОБЩИЕ СВЕДЕНИЯ}
	\addcontentsline{toc}{section}{ОБЩИЕ СВЕДЕНИЯ}
	В теории рассеяния на квантовых графах основным объектом исследования является одномерное стационарное уравнение Шрёдингера
   \begin{equation*}
   -\frac{\hbar^2}{2m}\frac{d^2\psi}{d x^2}+v\psi=E\psi,
   \end{equation*}
   Заменив $2E/\hbar^2=k^2$ и $2mv/\hbar^2=u$, получим уравнение в безразмерной форме
   \begin{equation}\label{Schred1D}
   -\frac{d^2\psi}{d x^2}+u\psi=k^2\psi.
   \end{equation}
   
   \begin{Def}
   	Оператор Шрёдингера ~--- это дифференциальный оператор \newline  $-\frac{d^2}{d x^2}+u$.
   \end{Def}

   \begin{Def}
   	Потенциал уравнения (потенциал оператора Шрёдингера) ~--- это коэффициент $u$.
   \end{Def}

   \begin{Def}
    Спектральный параметр ~--- это константа $k$.
   \end{Def}
	Отметим, что при $k=0$, получаем описание движения частицы в отсутствие внешних полей
	\begin{equation*}
	\psi=c_1e^{ikx}+c_2e^{-ikx}, \quad \text{если} \quad u\equiv0.
	\end{equation*}
    Также будем называть волновой функцией решение уравнения Шрёдингера.
    
    Волновая функция обладает следующими свойствами \cite{Peisakhovich, Landau}:
    \begin{itemize}
    	\item $\psi\left(x\right)$ ~--- комплекснозначная функция
    	\item $\psi\left(x\right)$ ~--- конечная функция (интегрируема с квадратом)
    	\item $\psi\left(x\right)$ ~--- однозначная функция координат и времени
    	\item $\psi\left(x\right)$ ~--- непрерывная функция
    \end{itemize}

   При этом физический смысл имеет только квадрат модуля волновой функции: \linebreak $\left|\psi\left(x\right)\right|^2 = \psi\left(x\right) \psi\left(x\right)^*$ ~--- плотность вероятности нахождения частицы в некоторой области.
   
   В теории рассеяния на квантовых графах часто налагают дополнительные ограничения на потенциалы: они считаются измеримыми функциями с конечными первыми моментами
   
   \begin{equation*}
   \int_{-\infty}^{\infty}(1+|x|)|u(x)|dx<\infty.
   \end{equation*}
   
   Это условие означает, что потенциал $u$
   достаточно быстро (быстрее чем $x^{-\delta}$, $\delta\geq2$)
   убывает при $|x|\rightarrow\pm\infty$.
   
   Переходя к асимптотике, получим:
   $$
   -\psi''=k^2\psi, \quad |x|\gg1,
   $$

   откуда имеем
   
   $$
   \psi=c_1e^{ikx}+c_2e^{-ikx}+o(1), \quad x\rightarrow\pm\infty.
   $$
   
   Без ограничений общности, будем считать, что источник волны находится в $+\infty$. Это соответствует следующему начальному условию на $-\infty$:
   \begin{equation}\label{SchredBorderMInf}
   \psi=e^{-ikx}+o(1), \quad x\rightarrow-\infty,
   \end{equation}
   Тогда асимптотический вид волновой функции на $+\infty$ имеет вид:
   \begin{equation}\label{SchredBorderPInf}
   \psi=a(k)e^{-ikx}+b(k)e^{ikx}+o(1), \quad
   x\rightarrow+\infty.
   \end{equation}
   
   \begin{Def}
   	Коэффициент прохождения ~--- это коэффициент $a = a\left(k\right)$, соответствует частице, движущейся влево.
   \end{Def}

   \begin{Def}
   	Коэффициент отражения ~--- это коэффициент $b = b\left(k\right)$, соответствует частице, движущейся вправо.
   \end{Def}

   \begin{Def}
   Данные рассеяния ~--- это набор всех коэффициентов отражения и прохождения.
   \end{Def}

   \begin{Def}
   	Прямая задача рассеяния ~--- вычисление данных рассеяния по заданному потенциалу согласно формулам (\ref{Schred1D}), (\ref{SchredBorderMInf}), (\ref{SchredBorderPInf}).
   \end{Def}

   \begin{Def}
   	Обратная задача рассеяния ~--- вычисление потенциала по заданным данным рассеяния согласно формулам (\ref{Schred1D}), (\ref{SchredBorderMInf}), (\ref{SchredBorderPInf}).
   \end{Def}

   В теории квантовых графов данные рассеяния обычно представляются в виде матрицы рассеяния. Так, для одномерного стационарного уравнения Шрёдингера она определяется асимптотиками:
   
   TODO: похоже, что тут надо поменять $s_{12} \leftrightarrow s_{11}, \quad s_{21} \leftrightarrow s_{22}$ .
   \begin{gather*}
   \begin{aligned}
   \psi_1(x,k)&=e^{ikx}+s_{12}(k)e^{-ikx}+o(1), \quad &x\rightarrow-\infty,\\
   \psi_1(x,k)&=s_{11}(k)e^{ikx}+o(1), \quad &x\rightarrow+\infty,\\
   \psi_2(x,k)&=e^{-ikx}+s_{21}(k)e^{ikx}+o(1), \quad &x\rightarrow+\infty,\\
   \psi_2(x,k)&=s_{22}(k)e^{-ikx}+o(1), \quad &x\rightarrow-\infty.
   \end{aligned}
   \end{gather*}
   

   \begin{Def}
   	S-матрица (матрица рассеяния) ~--- это матрица 
    \begin{equation}\label{DSM}
    S(k)=\left(%
    \begin{array}{cc}
    s_{11}(k) & s_{12}(k) \\
    s_{21}(k) & s_{22}(k) \\
    \end{array}%
    \right)
    \end{equation}
   \end{Def}

   Л.~Д.~Фаддеевым доказано \cite{SMatrix}, что существуют решения с заданными асимптотиками, а также S-матрица унитарна и удовлетворяет условию симметрии $s_{11}\left(k\right) = s_{22}\left(k\right)$.
   
   \subsection{Константный потенциал}
   Предположим, что в уравнении (\ref{Schred1D}) потенциал $$u = u\left(x\right) \equiv \tilde{u} = const \in \mathbb{C},$$ тогда имеем линейное однородное дифференциальное уравнение второго порядка с постоянными коэффициентами. Его характеристическое уравнение имеет вид
   \[-\lambda^2+\tilde{u} = k^2\]
   и корни
   \[\lambda_{1,2}=\pm \sqrt{\tilde{u}-k^2}\]
   Тогда общее решение имеет вид
   \[
   \psi\left(x\right)= C_1 e^{\sqrt{\tilde{u}-k^2}x} + C_2 e^{-\sqrt{\tilde{u}-k^2}x}
   \]
   
   В частности, если $\tilde{u} \in \mathbb{R}, \ \tilde{u}-k^2<0$, то, обозначив $-a^2=\tilde{u}-k^2$, решение можно записать в виде \[\psi\left(x\right)= C_1 \cos\left(ax\right)+ C_2 \sin\left(ax\right)\].
   
   \section{Задача Штурма-Лиувилля на компактных графах}

   Рассмотрим произвольный взвешенный граф с рёбрами конечной длины и семейство операторов $\{L_i\}_{i=1}^B$

   \begin{equation}\label{LOp}
   L_i = -\frac{d^2}{dx_i^2} + u_i(x_i).
   \end{equation}

   Каждому ребру $b$ сопоставим отрезок $[0, l_b]$, где $l_b$ --- длина ребра.
   Определим следующие пространства:
   \begin{gather*}
   D(L_i)={\varphi(x_i):\varphi\in C_0^\infty[0,l_i]},\\
   L_2(G)=\sum_{i=1}^{B}\oplus L_2[0, l_i],\\
   D(G)=\sum_{i=1}^{B}\oplus C_0^\infty[0,l_i],\qquad D(G)\subset L_2(G)
  \end{gather*}

  и оператор
  \begin{equation}\label{LG}
  L_G=\sum_{i=1}^{B}\oplus L_i.
  \end{equation}

  Поставим условия в вершинах графа следующим образом:

  \begin{enumerate}\label{Kirchoff}
	\item $\psi$ непрерывна в узлах графа,
	\item сумма производных волновой функции по всем дугам с учётом ориентации рёбер в каждой вершине равна нулю.
  \end{enumerate}

  Данные граничные условия называют законами Кирхгофа для квантовых графов \cite{KirchhoffRule}. Они имеют физический смысл: первое условие соответствует непрерывности волновой функции в вершине, второе условие соответствует сохранению потока в вершине.

Имеют место следующие теоремы \cite{GerasimenkoPavlov}:

   \begin{theorem}
	Дифференциальное выражение $L_G$ и граничные условия 1) и 2)
	определяют самосопряженный в $L_2(G)$ оператор.
   \end{theorem}

  \begin{theorem}
	Спектр оператора $L_{G}$ на компактном графе с локально измеримым
	потенциалом дискретен.
  \end{theorem}

  \begin{Def}
  	Задача Штурма-Лиувилля на компактном графе $G$ ~--- это задача поиска спектра оператора $L_{G}$.
  \end{Def}

  \begin{Def}
  	Волновая функция $\psi$ ~--- это $B$-компонентный вектор $$\psi=\left(\psi_{b_1}(x_{b_1}),
  	\psi_{b_2}(x_{b_2}),\ldots,\psi_{b_B}(x_{b_B})\right)^T,$$
  	
  	где множество $\left\{b_i\right\}_{i=1}^{B}$ состоит из различных
  	ребер графа $G$.
  \end{Def}

  На каждом ребре $b$ компонента волновой функции $\psi_b(x_b)$
  удовлетворяет одномерному стационарному уравнению Шрёдингера:

  \begin{equation}\label{Shred1}
   \left(-\frac{d^2}{dx_i^2} +
    u_i(x_i)\right)\psi_b(x_{i})=k^2\psi_b(x_{i})\qquad(b=[i,j]),
  \end{equation}

  а в каждой вершине -- граничным условиям 1) и 2). Заметим, что
  введённая ранее ориентация на ребрах не является существенным
  требованием, и служит лишь для удобства вычисления производных по
исходящим дугам.
   
   
\newpage.
\addcontentsline{toc}{section}{ЛИТЕРАТУРА}
\renewcommand{\refname}{ЛИТЕРАТУРА}
\begin{thebibliography}{1} % Ссылки на литературу приводятся только на английском языке (даже если у источника нет англоязычной версии).
	
	\bibitem{TransparentQuantumGraphs} J.~R.~Yusupov, K.~K.~Sabirov, M.~Ehrhardt, D.~U.~Matrasulov, {\it Transparent Quantum Graphs}, Phys. Lett. A 383, 2382 (2019).
	
	\bibitem{KirchhoffRule} V.~Kostrykin, R.~Schrader, {\it Kirchhoff's Rule for Quantum Wires} J. Phys. A: Math. Gen. Vol.32, (1999), 595--630.
	
	\bibitem{Kuhn} H.~Kuhn, {\it A Quantum-Mechanical Theory of Light Absorption of Organic Dyes and Similar Compounds}, J. Chem. Phys. \textbf{17} (1949), https://doi.org/10.1063/1.1747143.
	
	\bibitem{Superconductivity} S.~Alexander, {\it Superconductivity of networks. A percolation approach to the effects of disorder}, Phys. Rev. B \textbf{27}, 1541 (1983).
	
	\bibitem{Kowal} D.~Kowal, U.~Sivan, O.~Entin-Wohlman, Y.~Imry, {\it Transmission through multiply-connected wire systems}, Phys. Rev. B \textbf{42}, 9009 (1990).

	\bibitem{QuantumChaos} T.~Kottos, U.~Smilansky, {\it Quantum chaos on graphs}, Phys. Rev. Lett. \textbf{79}, 4794--7 (1997).
	
	\bibitem{SpectralEstimates} A.~Kostenko, N.~Nicolussi, {\it Spectral estimates for infinite quantum graphs}, Calc. Var. \textbf{58}, 15 (2019), Spectral estimates for infinite quantum graphs.
	
	\bibitem{deltaCouplings}  P.~Kurasov, A.~Serio, {\it Optimal Potentials for Quantum Graphs}, Ann. Henri Poincare \textbf{20}, 1517--1542 (2019). https://doi.org/10.1007/s00023-019-00783-6
	
	\bibitem{Popov1} D.~A.~Eremin, E.~N.~Grishanov, D.~S.~Nikiforov, I.~Y.~Popov, {\it Wave dynamics on time-depending graph with Aharonov-Bohm ring}, Наносистемы: физика, химия, математика, \textbf{9}:4 (2018), 457--463. https://doi.org/10.17586/2220-8054-2018-9-4-457-463
	
	\bibitem{Popov2} A.~A.~Boitsev, I.~Y.~Popov, {\it A model of an electron in a quantum graph interacting with a two-level system}, Наносистемы: физика, химия, математика, \textbf{10}:2 (2019), 131--140. https://doi.org/10.17586/2220-8054-2019-10-2-131-140
	
	\bibitem{GerasimenkoPavlov} Н.~И.~Герасименко, Б.~С.~Павлов, {\it Задача рассеяния на некомпактных графах}, ТМФ, \textbf{74}:3 (1988), 345 -- 359.
	
	\bibitem{SpectralSurgery} А.~Н.~Бондаренко, В.~А.~Дедок, {\it Спектральная хирургия квантовых графов}, Сиб. журн. индустр. матем., \textbf{7}:4 (2004), 16--28.
	
	\bibitem{Peisakhovich} Ю.~Г.~Пейсахович, А.~А.~Штыгашевич, {\it Одномерная квантовая механика}, монография, Новосибирск, Изд-во НГТУ, 2007.
	
	\bibitem{Landau} Л.~Д.~Ландау, Е.~М.~Лифшиц, {\it Квантовая механика (нерелятивистская теория)}, Издание 4-е, М.:Наука 1989, 768 с. - (<<Теоретическая физика>>, том \RomanNumeralCaps{3}) - ISBN 5-02-014421-5.
	
	\bibitem{SMatrix} Л.~Д.~Фаддеев, {\it Свойства S-матрицы одномерного уравнения Шрёдингера}, Труды МИАН, 1964, Т.73, с. 314--336.
	
	
	
\end{thebibliography}
\end{document}
